\section{Research Questions}
\textbf{RQ1} is the most well formed research question proposed in this work. It comes after much research into the application of temporal recognition to IR tasks. You can read more about this in \S \ref{background_survey}. Several approached are discussed and explained, however we must, through careful comparison and evaluation, decide which methods most directly apply to this sensitivity review problem.

\textbf{RQ2} ..

\textbf{RQ3} Through analysis and evaluation of the first two research questions we hope to acquire some insights into what makes a query good or bad in this task, and what features seem to lead to success. If the opportunity presents itself we can improve upon this

The problem this research seeks to address can be summarized in one question: "Can we apply a combination of state of the art IR techniques to improve the performance of an automatic sensitivity review system." There are a few key areas encompassed within "state of the art IR techniques" that will be specifically relevant. This project will seek to find these and prove their effectiveness. Specifically we will aim to produce evaluation results proving these claims of improved performance through the application of these techniques. The evaluation and discussion sections of the aforementioned previous project provide some starting ideas for what can be done to tackle this problem. These ideas are looked at extensively in \S \ref{background_survey}, but include ideas such as key phrase extraction, temporal tagging and learning to rank methods.

\section{Contributions}
This research will contribute a novel application of state of the art information retrieval techniques to a real world problem. It will also contribute to two areas of information retrieval which seem to be lacking in research - query generation from documents and making use of temporal information present in source and target documents.

\begin{table}[H]\label{table:time-allocation}
\begin{center}
\begin{tabular}{r|l}
    \emph{Section} & \emph{Intended completion time}\\
    Implement Temporal Systems & Late December\\
    Evaluate Temporal Systems & Mid January\\
    Investigate Other Techniques & February\\
    Evaluate Other Techniques & Early March\\
    Report Findings & End March\\
\end{tabular}\par
    \caption{Proposed Time Line for Completion of Work}
\end{center}
\end{table}