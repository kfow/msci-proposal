\documentclass{mprop}
\usepackage{graphicx}
\usepackage[colorinlistoftodos]{todonotes}

% alternative font if you prefer
% comment this out to go back
\usepackage{times}

% for alternative page numbering use the following package
% and see documentation for commands
%\usepackage{fancyheadings}


% other potentially useful packages
%\uspackage{amssymb,amsmath}
%\usepackage{fancyvrb}
%\usepackage[final]{pdfpages}
\usepackage{enumerate}
\usepackage{enumitem}
\usepackage[colorlinks=true, allcolors=blue]{hyperref}
\usepackage[colorinlistoftodos]{todonotes}
\usepackage{listings}
\usepackage[newfloat]{minted}
\usemintedstyle{emacs}
\usepackage{pgfplots}
\usepackage{enumerate}
\usepackage{enumitem}
\usepackage{url}
\usepackage{amssymb}
\usepackage{amsmath}
\usepackage{caption}
\usepackage{adjustbox}
\usepackage{subcaption}
\usepackage{inconsolata}
\usepackage{graphics}
\pgfplotsset{compat=1.8}
\usepackage{tabu}
\newcommand{\code}[1]{\texttt{#1}}
\newenvironment{codelisting}{\captionsetup{type=listing}}{}
\SetupFloatingEnvironment{listing}{name=Code Sample}

% Use “\cite{NEEDED}” to get Wikipedia-style “citation needed” in document
\usepackage{ifthen}
\let\oldcite=\cite
\renewcommand\cite[1]{\ifthenelse{\equal{#1}{NEEDED}}{\ensuremath{^\texttt{[citation~needed]}}}{\oldcite{#1}}}

\usepackage{xcolor,colortbl}
% A package which allows simple repetition counts, and some useful commands
\usepackage{forloop}
\newcounter{loopcntr}
\newcommand{\rpt}[2][1]{%
  \forloop{loopcntr}{0}{\value{loopcntr}<#1}{#2}%
}
\newcommand{\on}[1][1]{
  \forloop{loopcntr}{0}{\value{loopcntr}<#1}{&\cellcolor{gray}}
}
\newcommand{\onx}[1][1]{
  \forloop{loopcntr}{0}{\value{loopcntr}<#1}{&\cellcolor{orange}}
}
\newcommand{\off}[1][1]{
  \forloop{loopcntr}{0}{\value{loopcntr}<#1}{&}
}

\definecolor{orange}{HTML}{FF7F00}



\renewcommand{\arraystretch}{1.5}

\begin{filecontents*}{data.csv}
name map length time
allterms 	0.4554	358.5  	2.51435
ne 			0.3979  93.45  	0.4132
tfidf 		0.3038  10  	0.13675
subject 	0.184  	9.75  	0.1578
\end{filecontents*}

\begin{document}

%%%%%%%%%%%%%%%%%%%%%%%%%%%%%%%%%%%%%%%%%%%%%%%%%%%%%%%%%%%%%%%%%%%
\title{Research Proposal: Smart Sensitivity Review}
\author{Kelvin Fowler}
\date{\today}
\maketitle
%%%%%%%%%%%%%%%%%%%%%%%%%%%%%%%%%%%%%%%%%%%%%%%%%%%%%%%%%%%%%%%%%%%

%%%%%%%%%%%%%%%%%%%%%%%%%%%%%%%%%%%%%%%%%%%%%%%%%%%%%%%%%%%%%%%%%%%
\tableofcontents
\newpage
%%%%%%%%%%%%%%%%%%%%%%%%%%%%%%%%%%%%%%%%%%%%%%%%%%%%%%%%%%%%%%%%%%%

%%%%%%%%%%%%%%%%%%%%%%%%%%%%%%%%%%%%%%%%%%%%%%%%%%%%%%%%%%%%%%%%%%%
\section{Introduction}\label{intro}
% REWRITE INTRODUCTION TO GIVE MORE CONTEXT AND BE MORE CLEAR.
The National Archives are subject to the Freedom of Information Act (2000)~\cite{foi}. 
This means that they must respond to freedom of information requests and release documents after a certain amount of time.
There are many reasons why a document may be withheld~\footnote{https://ico.org.uk/for-organisations/guide-to-freedom-of-information/refusing-a-request/} \footnote{http://www.legislation.gov.uk/ukpga/2000/36/part/II}, \cite{foiexemptions}. especially if the document is deemed to contain sensitivities.
For example, a document may contain the name of confidential informants whose lives would be in danger if the document were released to the public.
For this reason there is a rigorous review process in place at The National Archives. Archivists must review documents and decide if they contain sensitivities.
One aspect of this review process is the comparison of document contents to information in the public domain, such as news articles.
If there is potentially sensitive information within a document, but the information is already known to the public, then the document can be released without cause for concern.
This is not a trivial task and it currently requires the archivists to use public search engines to query the public domain.
Not only is this slow, it is insecure and releases potentially sensitive information to an external service.
Further, with the recent mass adoption of digital documents, more documents must be kept than ever before as every email and digital document is automatically archived.
The sensitivity review process is struggling to keep up with the rate of document generation~\cite{allan2014record} and there is an increasing burden on archive facilities to decide ``what to keep''~\cite{moss2012have} as the collections of documents grow in size each day.

This abundance of digital documents presents an opportunity to adjust the sensitivity review process and introduce assistive technology to make the process easier.
Generally, this type of classification task falls into the domain of Information Retrieval (IR), which is the process of providing information relevant to a given information need.
Some work has been done to apply IR in the automatic classification of documents as sensitive (see Section~\ref{background_survey.sensitivity_review}), however this proposal seeks to address the identification of public domain knowledge task described above.
The information need is public domain documents relating to the contents of a given document which is to be reviewed.
If these public domain documents (such as news articles) can be automatically presented to the reviewer alongside the document for review, this would significantly decrease the workload and security concerns attached to this part of the process.
We do not seek to outright replace the role of manual review with technology, and it has been noted in the past that archivists are reluctant to trust technology alone in the sensitivity review process~\cite{gollins2014using}.
%The digitization of sensitivity review presents an opportunity for novel technology based approaches to ease the burden. Information Retrieval (IR) is the process of providing information relevant to a given information need. \textbf{IR naturally lends itself to this task.} The information need is \textit{public domain documents relating to a potentially sensitive document}. Indeed, IR has been applied in various ways to the task of sensitivity review (see Section~\ref{background_survey.sensitivity_review}. Sensitivity review is a classification task (that is, we want to specifically classify documents as \textit{sensitive} or \textit{not sensitive} and public domain knowledge identification is but one aspect of \textbf{this} process. Specifically, the identification of public domain knowledge aims to assist the manual review process, rather than to replace it. In fact, it has been noted that the manual review staff would be reluctant to allow technology to perform the task

This task of identifying public domain knowledge in relation to the contents of potentially sensitive documents has already been tackled in part by a Level 4 Project at the University of Glasgow. 
The system generated queries automatically from documents which were potentially sensitive and ran them through an IR search engine to return relevant public domain documents.
The system also provided a prototype user interface for presenting the results to manual reviewers.
The system proved successful, though during the the process of generating queries from potentially sensitive documents did not extend far beyond the extraction of named entities. 
These source documents contain other information which can be automatically extracted and included in a query.
This proposed research seeks to expand upon the system created last year in order to improve it's effectiveness and provide better public domain results for any given potentially sensitive document.
Some potential avenues for this expansion were identified during the evaluation stage of the L4 Project, such as extraction of temporal information and the use of machine learning.
We propose to use these techniques, and others, to generate further queries from potentially sensitive documents.
Further, we wish to investigate improvements to the entire retrieval system, in addition to the query generation phase.

%As a Level 4 project at the University of Glasgow, this task of identifying public domain knowledge was tackled in part. A Prototype UI and a REST API were produced. The system analyses documents to automatically generate queries to be run in a search engine in order to retrieve public domain documents. This project used natural language processing to tag named entities within the documents and generate queries from this data. This system of query generation was evaluated to find the best method. The evaluation led to conclusions and suggestions for directions of future work. These serve as an indication for the areas that we plan to investigate in this project.

% The project proposed in the document seeks to extend upon the system built last year. We aim to improve the performance of automatically generated queries significantly, through various techniques which will be discussed below. The me

%The project proposed in the document seeks to extend upon the system built last year in order to improve the query generation process. Once again, we will investigate the automatic formulation of queries from the contents of given documents which resemble those that are to be reviewed for sensitivities.

\subsection{Terminology}
% Is there enough context in the above introduction to justify having this here?
From henceforth a potentially sensitive document from which queries are generated will be called a \textbf{Source Document}.
In some cases, documents will be used which are not potentially sensitive, but are representative of the types of documents encountered during  sensitivity review.
A public domain document which is to be retrieved by the search engine will be called a \textbf{Target Document}.
Target Documents will be public domain documents such as news or Wikipedia articles.

Analogously, the set of all Source Documents will be referred to as the \textbf{Source Collection} and a set of Target Documents will be referred to as a \textbf{Target Collection}.
Potentially, the collection of all target documents is enormous, and consists of every public domain document. We will deal with some specific limited subsets of target documents.

The project which was completed last year will be referred to as the \textbf{L4 Project}.

\subsection{Structure}
This research proposal will take the following structure. 
In Section~\ref{problem_statement} we will more formally define the problem this research seeks to address, and how this research will contribute to the field of IR. 
In Section~\ref{background_survey} related literature is reviewed in order to investigate the proposed projects relevance and prompt discussion of potential avenues for this research to take. 
Section~\ref{proposed_approach} outlines the proposed approach the research will take, as well as detailing some preliminary investigations into the feasibility of these goals. 
This includes some discussion motivations for choices of technologies.
Finally, Section \ref{work_plan} highlights some proposed implementation details, deliverables and potential dates for their completion.

%%%%%%%%%%%%%%%%%%%%%%%%%%%%%%%%%%%%%%%%%%%%%%%%%%%%%%%%%%%%%%%%%%%
\section{Statement of Problem}\label{problem_statement}
With some motivation in mind for the problem this research might address, we must now consider how to formalize our aims. 
As will be discussed in more detail in Section~\ref{background_survey.previous_project} the L4 Project began by investigating the use of named entities present in source and target documents.
Named Entities are words or phrases (generally proper nouns) referring specifically to people, places or things.
These named entities were identified in both source and target documents so that they could be directly matched during retrieval. 
Time dictated that the approaches did not extend far beyond this, however the findings provided useful insights which have been used to formulate some of the ideas in this proposal. 
Some other extractable items in addition to named entities are present in the bodies of source documents such as dates, subjects, noun phrases, verb phrases and other n-grams. 
Generally, there exists reasonable methods for automatic extraction of many of these such as the Stanford NLP Tool-kit~\cite{manning2014stanford}. 
Further, the specification for the L4 Project originally mentioned making use of some machine learning techniques. 
This was not investigated due to time constraints, but machine learning (especially learning to rank) has been shown to improve performance in many retrieval tasks (see Section~\ref{background_survey.others}).
These methods will be explored in detail throughout Sections \ref{background_survey} and \ref{proposed_approach}. 
We wish to combine these ideas to create a performant retrieval system which can return target documents which are relevant to the content of a source document, in a way which is assistive to sensitivity review. 
%We wish to make use of relevant IR technologies outside of the previously explored query generation through named entity extraction and to discover what information a source document contains that is the most important for the task of retrieval of public domain knowledge.
Relevant IR technologies can be employed to extend upon the extraction of named entities, allowing us to formulate queries containing other sources of information.

\subsection{Research Questions}
The following research questions formalize the above discussion and describe the goals of this proposed project.
\begin{enumerate}[label=\textbf{RQ.\arabic*}]
\item \textit{What extractable information within the body of source documents (such as Named Entities, Temporal References and Other n-grams) are most important in the formation of a query for retrieval of public domain documents relevant to events and concepts mentioned in a source document? }

\textbf{RQ1} has already been tackled in part. 
The L4 Project showed that named entities extracted from source documents provide a good base query with which to perform retrieval. 
However it was also identified that by removing much of the other information and context from the source document we lost some search accuracy (see Section~\ref{background_survey.previous_project} for more detail). 
The task of comparison of public domain information in sensitivity review often deals with references to specific events. 
These events occur at a given time and so the temporal information present in source documents can be used to identify these time frames. 
We must also discover what other information contained in source documents can be leveraged to assist this task.
Context surrounding named and temporal entities could be important. 
The challenge remains to identify what constitutes context and how it can be leveraged.

\item \textit{Can a comprehensive retrieval model be produced to generate queries from unseen source documents to retrieve public domain documents related to events and notions inside the source document?}

\textbf{RQ2} addresses a potentially more difficult question. 
In \textbf{RQ1} we discuss the extraction of features from source documents such as temporal entities and named entities. 
We must form these extracted features in to a query and run it through some IR search engine. 
We must therefore create a supporting framework which allows for the seamless extraction of these features and the ability to use them in queries. 
Not only that, there is much more to IR than sound query formulation. 
Machine learning techniques such as learning to rank and query expansion techniques like pseudo-relevance feedback can be applied to improve performance. 
We must consider which techniques are most appropriate in the scenario of retrieval of public domain knowledge and effectively apply and integrate them. 
This entire system must be evaluated in order to answer \textbf{RQ2}.
\end{enumerate}

These research questions allow us freedom to investigate multiple avenues of query generation and retrieval techniques. At the end of Section~\ref{proposed_approach} we give more detailed hypotheses upon which to evaluate our proposed work.

% \subsection{Old Research Questions}
% \begin{enumerate}[label=\textbf{ORQ.\arabic*}]
% \item What is the most appropriate way to integrate and apply recognition of temporal entities into the query generation process and retrieval model, so as to improve the effectiveness of this search system?
% \item How can we use the machine learning to improve the effectiveness of our information retrieval system?
% \item Are there any other methods which measurably improve search effectiveness?
% \item Can this task be improved through identifying source document focus time?
% \item Can we directly compare source and target documents (including a temporal aspect) to retrieve target documents?
% \item Does including temporal data in query formulation improve search results?
% \item Does query expansion through knowledge base linking improve search results?
% \item Can Learning to Rank (L2R) techniques be used effectively in this problem domain?
% \item How else can we formulate queries from long documents?
% \end{enumerate}

\subsection{Scope of this Project}
We now define the expected scope of this proposed project. Specifically, this is what we propose to address, what we may address if time permits and what will not investigate at all.
Clearly, we must investigate, implement and evaluate a system which can answer the above research questions (\textbf{RQ1} \& \textbf{RQ2}). 
In order to do this, we must create a system which generates queries from documents, runs these queries against a target collection and analyses the quality of the results.

Assuming this initial research is done is satisfactory time, other avenues may be explored such as knowledge base linking, query expansion and resource selection. 
More detail on these concepts can be found in Section~\ref{background_survey}.

This proposed project will explicitly not focus on developing or improving any kind of user interface. 
This was a sizeable aspect of the L4 Project, however will not be in the scope of this proposed work.

\subsection{Contributions}
This research will contribute an improvement to the understanding of the formation of queries from source documents. As far as we know, it will be the first work of it's kind which seeks to combine the ideas of named entity extraction, temporal entity extraction and learning to rank in order to assist sensitivity review.
Specific techniques for query generation from documents will be described and empirically evaluated. 
This research has applications in many areas of information retrieval outside of sensitivity review (for example, finding related web pages, news stories or other automatic recommendations).

%%%%%%%%%%%%%%%%%%%%%%%%%%%%%%%%%%%%%%%%%%%%%%%%% BACKGROUND SURVEY/ LIT REVIEW
\section{Background Survey} \label{background_survey}
This section will present an overview of the various pieces of related literature to this proposed research. 
We will discuss previous work completed in the field of sensitivity review as well as other information retrieval studies which are applicable and relevant.

\subsection{Previous Project} \label{background_survey.previous_project}
This proposed project is hinged on the existence of the work completed last year as part of the L4 Project \cite{DissertationKelvinFowler}. 
The L4 Project was very much an investigatory project into the domain of using information retrieval to assist sensitivity review. 
Although the L4 Project focussed heavily on a user interface (work which will not continue in this proposed research) it also began to tackle the task of effective retrieval of target documents given some source document. 
The source document was used to generate a query which could be performed in the IR search engine Terrier, against the collection of target documents.
Retrieval using these generated queries was empirically evaluated in order to understand the performance of different query formulations. 
The evaluation results of the L4 Project form a baseline of performance that we wish to improve upon in this proposed research.
They also provide good indication of which avenues to proceed or ignore.

The implementation of the L4 Project can also be reused as a starting point for this proposed research. 
It features some custom indexing tokenizers which extract named entities from source documents. 
This can be expanded to extract the other information we are interested in (temporal entities and other chunks). 
This indexing step allows us to iterate through terms in a document, analyse them and store them in some way. 
The indexing step can be run on source and target documents alike in order to extract and tag the matching information in each.

After indexing the source documents and identifying the named entities, four queries were generated for use in retrieval. These were:
\begin{itemize}
  \item \textit{All Terms Query} - All terms, except stop words, were kept for this query. Named entities were identified in order to match the indexed target collection.
  \item \textit{Named Entities Query} - This query consisted of all the named entities in the document and nothing else. This meant the query was fairly short, however it did not contain any additional context. Named entities were extracted using Stanford Core NLP~\cite{manning2014stanford}. Persons, Locations and Organisations were extracted.
  \item \textit{Tf-Idf Named Entities Query} - The named entities were given a Tf-Idf ranking and the top ten were used a query. Tf-Idf is a commonly used IR technique which attempts to estimate term importance through measures of how frequently the term appears in a document compared to a collection of documents.
  \item \textit{Subject Query} - The subject line from a source document was used a query after having its named entities identified. This was an attempt to add context to the queries, as well as being short and easy to compute.
\end{itemize}

These queries were used as is in Terrier v.4 with very little tweaking of the underlying retrieval model. A more thorough investigation of how to improve the configuration of Terrier for this task is an important part of this proposal.

\subsubsection{Evaluation and Findings}
In order to evaluate the system, a test collection was formulated using a sample of source documents. 
These were wikileaks files which were identified by National Archive's reviewers to be representative of real documents which must be reviewed for sensitivities.
The target collection was a collection of Associated Press news articles.
In order to generate an effective test collection, 20 source documents were chosen and the 4 above queries were generated and run from each. 
The top 20 returned documents for each of these queries was then judged as relevant or non-relevant. 
This formed the test collection.
This was expensive to produce, and so we opt for a different method in this proposed research (see Section~\ref{proposed_approach} for details.)
These relevance judgements were then used to get results on the effectiveness of the various query formulations.
It was found that the performance of the \textit{All Terms Query} was the objectively the best, but the query took far too long to perform. 
The query processing time is a measure we must take into account in this proposed project as having an excellent performing query is pointless if it cannot be performed in a reasonable time, especially in the context of assistive review. The shorter \textit{Named Entities Query } was therefore chosen as the most effective considering all factors. Results can be seen in Table~\ref{standard_results}. Figure~\ref{mapgraphs} also shows the clear correlation between query length and processing time, as well as illustrating the point that a significantly shorter (and therefore faster) query can perform reasonably well.

\begin{center}
\begin{table}[h]
\centering
\resizebox{\columnwidth}{!}{%
\begin{tabular}{|c|c|c|c|c|c|}
\hline
Query					& MAP 				& Mean Query  		& Mean Query 			& Mean Reciprocal 		& Mean Precision 	\\ 
						& 					& Length (Words) 	& Processing Time (s) 	& Rank 					& at 4				\\	\hline
All Terms             	& \textbf{0.4554} 	& 358.5             & 2.51435               & \textbf{0.8500}       & \textbf{0.6750} 	\\	\hline
Named Entities 			& 0.3979 			& 93.45    			& 0.4132 				& 0.7526				& 0.5875 			\\	\hline
Tf-Idf Named Entities 	& 0.3038 			& 10                & \textbf{0.13675}      & 0.7236       			& 0.4375 			\\	\hline
Subject               	& 0.1840 			& \textbf{9.75}     & 0.1578                & 0.3367       			& 0.2500 			\\	\hline
\end{tabular}%
}
\caption{Results of Offline Evaluation, Adapted from Level 4 Project~\protect\cite{DissertationKelvinFowler}}
\label{standard_results}
\end{table}
\end{center}

\begin{figure}[H]
\centering
\begin{subfigure}[t]{.5\textwidth}
\centering
\begin{tikzpicture}
 \begin{axis}[
 	ylabel=$MAP$,
    xlabel={Mean Query Processing Time},
    width=0.95\linewidth,
    height=0.5\linewidth
    ]
        \addplot table[x=time,y=map] {data.csv};
    \end{axis}
\end{tikzpicture}
\end{subfigure}%
~
\begin{subfigure}[t]{.5\textwidth}
\centering
\begin{tikzpicture}
 \begin{axis}[
 	ylabel=$MAP$,
    xlabel={Mean Query Length},
    width=0.95\linewidth,
    height=0.5\linewidth
    ]
        \addplot table[x=length,y=map] {data.csv};
    \end{axis}
\end{tikzpicture}
\end{subfigure}
\caption{Analysis of Mean Average Precision (MAP) Scores against Query Processing Time and Query Length. Adapted from Level 4 Project~\protect\cite{DissertationKelvinFowler}} \label{mapgraphs}
\end{figure}

Other noteworthy findings include the fact that Named Entities of the Person type seem to have the biggest bearing on retrieval performance of the 3 named entity types used. 
Also, the extra context present in the All Terms Query clearly helped its performance. 
A key takeaway for this research is the challenge of how best to optimize performance while including enough context around named entities. Some qualitative investigation of the performance of certain queries yielded some additional insights. 
It was clear that although the system was able to match specific named entities, for example ``George Bush'' and ``Iraq'', there are many instances where these named entities could appear in a public domain document. A proposed way to deal with this ambiguity was inclusion of temporal data in order to remove irrelevant results from the wrong time frame.
Further, it was noted that the L4 Project did not make use of the popular process of learning to rank, which is almost ubiquitous in IR systems presently. 
These are both aspects which we propose to work on in the current proposed project.

\subsection{Sensitivity Review} \label{background_survey.sensitivity_review}
Assistive digital sensitivity review is an open challenge in information retrieval, and is being actively tackled, mainly by a team here at The University of Glasgow. 
Together with Timothy Gollins at The National Archives researchers at The University of Glasgow identified the open challenges in the application of IR to sensitivity review.
Most of the work in this topic has been in the realm of automatically classifying sensitivities in documents or sensitive documents. 
McDonald et al~\cite{mcdonald2014towards} experiment with creating a an automatic classification system for sensitivity review. This system uses features such as mentions of foreign countries and subjective sentiment in order to classify sensitivities relating to international relations.
In another investigation McDonald et al~\cite{mcdonald2015using} attempted to identify which parts of speech patterns most likely contained sensitivities relating to content supplied ``in confidence''.
More recently, McDonald et al~\cite{mcdonald2017enhancing} apply machine learning techniques (specifically semantic relationship recognition through word embeddings) to further assist the sensitivity classification process. 

Outside of the work conducted here at the University of Glasgow, Sanchez et al~\cite{sanchez2012detecting} discuss automatically identifying sensitivities in documents in order to sanitize these documents before release. 
They aim to remove personal information by inspecting noun phrases which contain more information than others. 
This is relevant to the work proposed here as the ability to recognise the most informative noun phrases could be used in order to formulate a highly effective query, by disregarding unhelpful terms and phrases.

These sensitivity review papers do not refer to the challenge of public domain knowledge comparison in sensitivity review, but instead tackle the other challenges it presents. 
While providing good background into the problem domain, they do not necessarily provide information on the specific task we propose. This proposed research does not seek to directly classify documents as sensitive or not sensitive, but rather provide a system which can assist manual sensitivity review. 
The comparison to public domain documents is a difficult and arduous manual task, and it has been noted that archivists are reluctant to trust technology entirely~\cite{gollins2014using}, although some intervention is needed if The National Archives are to keep up with the ever growing numbers of documents. This proposed work seeks to provide some of this assistance.


\subsection{Technology Assisted Review} \label{background_survey.tech_assisted_review}
More generally, sensitivity review falls into the category of Technology Assisted Review (TAR). 
TAR is technology which attempts to improve or assist a manual review process, such as sensitivity review, however some manual aspect must remain.
Other applications of technology assisted review are E-Discovery as described in this overview by Oard and Webber~\cite{oard2013information}, where the retrieval task is retrieving all relevant documents relating to the opposing party in a civil litigation case.
Grossman and Cormack~\cite{cormack2014evaluation} have pushed the capabilities of TAR in E-Discovery through investigation of various machine learning techniques. 
They focus on the application of continuous active learning. This is relevant to this proposed research as it involves a manual reviewer continuously examining suggestions made by the system. 
Interestingly, these manual reviews are added to the training set to further improve the retrieval of yet more documents. 
Not all techniques applicable to E-Discovery are relevant to this proposed research. E-Discovery is a high recall task, in that it is desirable to retrieve \textit{all} relevant documents for a given case. 
This research seeks a high precision retrieval scenario in that we want more relevant documents at higher ranks of the retrieval results.
This type of technology assisted review has also made headlines in recent years when the FBI used technology to assist their review of documents during the 2015 Hillary Clinton email controversy~\cite{cnnclinton}.


\subsection{Temporal Information Retrieval} \label{background_survey.temporal_ir}
Some work has been done in the field of incorporating temporal data into IR systems. 
In Section~\ref{proposed_approach} we discuss some ways in which temporal information could be applied to this proposed research.
In order to deal with temporal expressions in documents at all, we require a solid foundation to extract and normalize these expressions. Noteworthy temporal taggers include GUTime~\cite{mani2004recent} and more recently SUTime~\cite{chang2012sutime} and Heideltime~\cite{strotgen2010heideltime}. All of these taggers resolve temporal information in documents to the TimeML~\cite{pustejovsky2003timeml} TIMEX3 markup language, which defined a specific syntax for defining dates and times. This kind of temporal tagging technology is vital for any kind of use of temporal information from within documents.

In terms of actual applications of temporal information in IR, most relevant to this proposed research is the work of Berberich et al~\cite{berberich2010language} which deals specifically with temporal expressions in queries. 
Beyond basic matching of temporal information as text, Berberich et al~\cite{berberich2010language} use a language modelling approach to attempt to predict the likelihood of temporal expressions in a query being produced from a language model of a given document. Indeed, this begins to answer some of the open questions posed in a review of temporal information in IR~\cite{alonso2011temporal}. This is potential extension to the method we propose in Section~\ref{proposed_approach}.
Strötgen et al~\cite{strotgen2012identification} describe a method of identifying top relevant temporal expressions in documents. 
That is, identification of which temporal expressions are of most import in a document. 
One can also check which temporal expressions are of most import in relation to a given query (or corpus). 
This has clear applications to this research in two ways. The first being the ability to rank temporal expressions for the purpose of query generation. 
The second being query based features which can be used to retrieve relevant documents (those with more relevant temporal expressions are potentially more relevant).
Li and Croft display in~\cite{li2003time} a ``time based language model'' which promotes in search results documents from a given time period. This has the application in this proposed research of the ability to promote target documents which are from a similar period to the source document.
Jatowt et al~\cite{jatowt2013estimating} propose a method to estimate the time that a document focusses on. 

Most of these methods are complex and require a large supporting system to be built, and although the results look promising, recreating the implementation from the papers alone is a challenging task. As such they will remain as inspiration and may be incorporated if time permits following more rudimentary approaches to the use of temporal information.

\subsection{Complex Queries} \label{background_survey.complex_queries}
IR technology is continuously evolving and improving. One of these improvements is the idea of a complex or multi-part query.
This allows us to formulate a query of several distinct subqueries, each of which represents something different. 
Not only that, specific rules can be applied to sections of the query to provide the search engine with complex instructions to follow during retrieval.
Previously, in the L4 Project, we merely formed a query out of plain text and ran it through Terrier. 

Indri~\cite{strohman2005indri} is an existing search engine with this type of complex query feature, facilitated through the Indri Query Language~\cite{strohman2005indri}.
The Indri query language provides a format for representing queries with multiple probabilistic features and specific logical combinations. 
A precursor to Indri was the Inquery search engine, which provided the basis for complex queries in a probabilistic retrieval model~\cite{callan1992inquery}. 
We propose in this research to use this type of complex query structure in the formation of automatic queries from source documents. It allows one to separate classes of information extracted from the document and we will see in Section~\ref{proposed_approach.l2r} that this enables an interesting use of learning to rank.
Further, the rules present in a complex query system provide flexibility in searching and allow the developer to more robustly design effective queries than the generic verbose query approach used in the L4 Project.

As a concrete example, Lee et al~\cite{GeneratingQueriesLee12} use IndriQL when generating queries from arbitrary sizes of user selected text. This allows them to weight sections of queries more or less heavily depending on learned results.

\subsection{Learning to Rank} \label{background_survey.l2r}
Learning to rank is a well investigated application of machine learning to IR. 
It allows one to learn a weighting system for features of a document in order to rank it in the most effective way. 
This is done using a training set of existing relevance judgements. 
There are several works which cover this topic which are relevant to the work we propose in this document.
Liu et al~\cite{liu2009learning} presents an excellent overview of the topic with explanations of different methods of learning to rank.
Listwise learning to rank requires us to choose a specific effectiveness measure upon which to judge the effectiveness of the various features, this can be configured to any of the standard IR performance measures and so some investigation must take place in each distinct scenario to decipher which produces the best performance.
Learning to rank requires the use of sampling in order to be effective~\cite{macdonald2013whens}.
Sampling is the initial retrieval of a selection of documents which must then be re-ranked.
Sampling is used at two stages of the learning to rank process, when training the ranking model a sample of documents is obtained using an existing retrieval technique. 
The feature vectors for these documents are calculated and and retrieval measure is attempted to be maximised through re-ranking of these documents. 
This re-ranking will display which features of the documents are the most or least important. In fact, Dang et al~\cite{dang2013two} propose that we can actually view these sampling and re-ranking steps as two separate stages. A model can then be learned for the both the first stage and the second re-ranking stage. We seek to attain high recall in the first stage, so as to have as many relevant documents as possible to re-rank in the best possible order at the second stage. This presents another interesting avenue for this proposed research, whereby we may seek to discover which features we can identify which improve recall, for the first stage, and precision for the second.

Specifically, the LambdaMART approach to learning to rank has been shown to have excellent performance~\cite{macdonald2013whens} and a implementation of it won the 2010 Yahoo! learning to rank competition~\cite{chapelle2011yahoo}. The JForests~\footnote{http://code.google.com/p/jforests/} implementation of LambdaMART is distributed with Terrier, making it a very suitable choice for application in this proposed research.

\subsection{Other Uses of Machine Learning in IR} \label{background_survey.other_ml}
Aside from learning to rank, machine learning can be employed at other phases of the retrieval process. Most relevant to us are studies which investigate using learning to rank in order to better formulate queries.
Lee et al~\cite{GeneratingQueriesLee12} build on top of Xue et al~\cite{xue2010improving} in order to build an effective system for generating queries from arbitrary selected chunks of text. This uses a complicated machine learning technique which tries retrieval using different sections of the query in order to learn which are the most important. It requires relevance judgements to exist for the training phase, but does not require a gold standard training set of best chunks from a section of text. This is relevant to us in that it displays a way to extract important information from an arbitrary body of text. It is however an extremely complicated technique, which only covers generic ``chunks'' extracted from text.

Similarly, Bendersky et al~\cite{bendersky2010learning} learn the specific importance of terms within a query by measuring various features for parts of the query. These include measures like how many times it appears a Wikipedia title and it's frequency of appearance in MSN search logs. 

Both of these papers present techniques which could be applied to this research in order to discover which parts of a source document are most important, or contribute the best to a query. They are, however, complicated to implement and so would only be included after a more simple method is implemented and evaluated, in order that the more general retrieval system is shown to be effective.

\subsection{Knowledge Base Linking and Query Expansion} \label{background_survey.knowledge_base}
Query expansion is a commonly used IR technique which allows one to extend an existing query with additional terms. Often this is done using Pseudo-Relevance Feedback, which involves extracting terms from the set of top retrieved documents from a query and adding them to the query to return yet more results. This has been well examined in papers such as~\cite{cao2008selecting,yu2003improving}. This is another potential avenue of exploration for our query generation phase, although it is again something to be considered later in the project.

Another method which has proven effective for query expansion is knowledge base linking. Knowledge bases are large collections of information which are searchable. They contain named entities, events, dates and other terms along with additional information. They can generally be accessed programmatically. Dalton et al~\cite{dalton2014entity} exemplify the use of this technique through extracting entities and linking them to knowledge base information to expand queries. They show it has considerable performance improvements over existing query expansion techniques. This is a potential avenue of expansion for this proposed research, but will only be pursued following the implementation and evaluation of the more well defined proposals, see Section~\ref{proposed_approach}.

\subsection{Others} \label{background_survey.others}
The end goal is a system which can identify related public domain documents for a source document from the set of all public domain documents. This means retrieval will have to take place on many, disparate target document collections (or potentially even a web search). Si et al~\cite{si2002language} propose a method to effectively select the best resource collection for retrieval based on a query. Although it is unlikely we will deal with this problem in this research it is worth noting that this is an issue which has some existing research. Although this type of work is outside the scope of the project currently, the ideas could be valuable for future research.

% TODO: Could you mention information you found in IR Text books --- Not a terrible idea.

%%%%%%%%%%%%%%%%%%%%%%%%%%%%%%%%%%%%%%%%%%%%%%%%%%%%%%%%%%%%%%%%%%%
\section{Proposed Approach}\label{proposed_approach}
This section will describe the techniques and methods we propose to use in order to research the described problem.
We draw inspiration from the works discussed in Section~\ref{background_survey}. 
We wish to give a high level view of the various IR techniques than can be employed and how they can be evaluated.

A basis exists already for this project in the form of the software created in the L4 Project. 
We once again plan to use the Terrier~\cite{macdonald2012puppy} search engine for the IR workload of this project. 
Terrier is developed and maintained at the University of Glasgow, so the local knowledge is invaluable. 
The proposed project developer also has extensive background with Terrier and so it makes sense to continue to use this search engine. 
We must however build a system on top of Terrier to perform the IR task at hand. 
Terrier does not have the facility which allows a full source document to be efficiently provided as a query.
It has a select number of tokenisers which extract terms from target documents and add them to an inverted index, however we must add custom terms to an inverted index in order to match on specific named entities and temporal references. In the subsequent sections we will discuss the various facets of this proposed system and why we intend to do this, as well as some investigation of feasibility and some preliminary testing. To conclude we discuss our proposed methods of evaluation.

\subsection{Improved Named Entity Extraction}
The most obvious and immediate thing to do would be to try and improve the named entity extraction techniques which exist already.
The software produced for the L4 Project tags unigrams as named entities. For example given the name ``Jerry Seinfeld'', the system identifies that ``Jerry'' is a named entity as well as ``Seinfeld''. These are added separately to the list of named entities and there is no indication that they have any relationship. A improvement to this would be to explicitly categorise the bigram ``Jerry Seinfeld'' as a single named entity. We therefore retain the initial intended full name relationship.
The L4 Project retrieval system, however does not contain any logic for dealing with this bigram named entity. A system of complex query formulation is discussed in Section~\ref{proposedapproach.complexquery} which seeks to deal with this issue.
As in the L4 Project, named entity extraction will be performed using Stanford CoreNLP~\cite{manning2014stanford}. The documentation is easy to understand and the proposed project developer has experience using it from the L4 Project. The configuration requires some tweaking to achieve the above suggested improvement.

\subsection{Temporal Tagging and Matching}
Temporal information is another class of information which can be extracted from within documents. There are often references to specific times throughout the body of a document. These are not always immediately obvious to resolve to a specific time, and often require the knowledge of the document creation date to understand fully. There are similar technologies to the named entity tagger which allow for this type of temporal resolution. If we are able to extract these times in a standardised format we can use them during retrieval as parts of queries. The temporal tagging must also be run on the target collection in order to ensure Terrier has terms to match upon.
Another feature to easily extract is specific temporal information. 

There are several temporal taggers in existence which could be integrated into the system. 
Since the project currently uses StanfordNLP~\cite{manning2014stanford} we can easily integrate SUTime~\cite{chang2012sutime}, the temporal tagger included in StanfordNLP into the project.
There are several other choices of temporal taggers. 
Notably we have Heideltime~\cite{TemporalKuzeyEtAl2016}. However these are compared in~\cite{chang2012sutime} which shows SUTime to be the best performer. This is not a consistent conclusion throughout the literature with some studies indicating very close or equivalent performance, such as UzZaman et al~\cite{uzzaman2012tempeval}.
It is important however to check that this is true for our own needs also. 
As such, some rudimentary tests were performed on to compare the effectiveness on date resolution of Heideltime and SUTime. This serves the dual purpose of selecting the correct temporal tagger and ensuring that our proposed approach is feasible for representative documents.

\subsubsection{Temporal Tagging Comparison}
In this section we present our comparison of SUTime and Heideltime through an evaluation of their effectiveness on 10 documents which contain relative references to time (e.g. ``\textit{yesterday}'', ``\textit{last month}'').

In the comparison, the output of both taggers was reviewed manually. We wished to compare the taggers on their ability to specifically identify discrete time references in text. 
To clarify sometimes terms such as \textit{``currently''} resolved to a \code{PRESENT\_REF}. 
This is not particularly insightful or helpful, as it is not specific enough to match times between source and target documents and so examples such as this were completely ignored. 
Further, both SUTime and Heideltime contain facilities to identify sets or ranges of times. 
Again these were ignored in favour of explicit references, due to this being the initial feature we wished to match between source documents and target documents. 
Perhaps if time permits in this project we may implement a system to allow comparison of time ranges. 
An example of one of these specific references is: 
\\ \code{<TIMEX3 tid=``t1'' type=``DATE'' value=``2008-05-01''>Thursday</TIMEX3>}.
This document was created on \code{2008-05-01}, hence the specific resolution of ``Thursday'' to the observed date.
In the manual review we identified True Positives, False Positives, True Negatives and False Negatives. True Positives were correctly identified specific times. False Positives were when explicit times were identified incorrectly. This can be seen in the following example. ``The Board of Directors of SNEPCI elected Viviane Zunon Kipre as Chairperson on \textbf{January 28} and RTI's Board elected Honore Koffi Guie as its Chairperson \textbf{the same day}.''\footnote{Document No, Collection Name} To a human reading this text, we know that ``the same day'' should be resolved to January 28th, as it is previously mentioned in the sentence. This is one example where the behaviour of SUTime and Heideltime differ, with SUTime resolving it to the document creation date and Heideltime outputting the correct resolution. True Negatives were all terms which did not refer to dates that the taggers correctly skipped. False Negatives were when the tagger missed an explicit reference to time (e.g. ``2007/08'' being missed due to the ambiguity of the slash.)
Both taggers resolve to TimeML's TimeX3~\cite{timeml} mark-up language which gives a precise format to time annotation.
When configuring Heideltime the ``news'' style was used throughout.

Both taggers required minimal effort to integrate with our existing system. In fact, SUTime was included as part of the dependencies required for our existing system of named entity recognition. The Accuracy and the Standard F-Measure (\textbf{$ F_1 $}) were calculated and are reported below in Table~\ref{temporalcomparison}.

\todo{mcnemars test}

\begin{table}[H]
\centering
\begin{tabular}{|c|c|c|c|c|c|c|c|c|}
\hline
& \multicolumn{4}{|c|}{SUTime}    & \multicolumn{4}{|c|}{Heideltime} \\ 
\cline{2-9}
Document & TP  & FN & FP & TN     & TP & FN & FP & TN    \\ \hline
1        & 11  & 0  & 1  & 374	  & 12 & 0  & 0  & 374   \\ \hline
2        & 5   & 0  & 0	 & 788 	  & 5  & 0  & 0  & 788   \\ \hline
3		 & 3   & 0  & 0  & 228	  & 3  & 0  & 0  & 228 	 \\ \hline
4        & 1   & 0  & 1	 & 70  	  & 2  & 0  & 0  & 70  	 \\ \hline
5	     & 2   & 0  & 0	 & 64     & 2  & 0  & 0  & 64    \\ \hline
6		 & 14  & 2  & 0	 & 285 	  & 15 & 1  & 0  & 285   \\ \hline
7		 & 12  & 0  & 0	 & 389 	  & 11 & 1  & 0  & 390 	 \\ \hline
8		 & 2   & 0  & 0	 & 71  	  & 2  & 0  & 0  & 71  	 \\ \hline
9		 & 9   & 0  & 0	 & 287 	  & 9  & 0  & 0  & 287 	 \\ \hline
10		 & 7   & 0  & 1	 & 272 	  & 8  & 0  & 0  & 271 	 \\ \hline
Total    & 66  & 2  & 3	 & 2828	  & 69 & 2  & 0  & 2828  \\ \hline
F1-Score & \multicolumn{4}{|c|}{0.9635} & \multicolumn{4}{|c|}{0.9928} \\ \hline
Accuracy & \multicolumn{4}{|c|}{0.9982} & \multicolumn{4}{|c|}{0.9997} \\ \hline
\end{tabular}
\caption{Heideltime vs. SUTime on Sample Source and Target Documents}
\label{temporalcomparison}
\end{table}
Both display high measures of effectiveness, however Heideltime appears to be marginally better for our purposes. 
Through the course of the manual review of the data this conclusion can be confirmed. 
Heideltime seemed more close to way a human would identify dates (for example, ``the same day'' above). 
The date ranges that Heideltime identifies could also be helpful in the future of this project. 
Due to these results, we propose to use Heideltime as the temporal tagger to extract dates for query generation.

\begin{table}[H]
\centering
\begin{tabular}{|c|c|c|c|c|c|c|c|c|}
\hline
Reference      & SUTime          & Heideltime  & Correct                \\ \hline
2007/08		   & Missed Both     & 2007        & 2007 \textbf{and} 2008 \\ \hline
'88 Generation & 1988            & Missed 	   & 1988  	   	            \\ \hline
\end{tabular}
\caption{False Positives}
\label{temporalcomparison}
\end{table}

\subsubsection{Use of Temporal Tagging}
Although we plan to use Heideltime for temporal tagging of source and target documents, we must now understand how we can incorporate this into the retrieval model.
As discussed in \ref{background_survey} there are several existing approaches using temporal matching that we could implement and extend.
We propose, at least initially, to represent temporal entities textually as a string. 
In the retrieval model we can then match exactly these temporal entities in source and target documents as if they were any other term.
If this proves successful and time allows, we may look to extending the use of this temporal tagging. 
Makkonen et al~\cite{makkonen2004simple} describe a temporal similarity vector approach, which they concede could use improvement, although it could provide a baseline implementation of more complex temporal comparison.
Further, another interesting method appears in~\cite{jatowt2013estimating}, where Jatowt et al propose a method for estimating document focus time. 
Combined with the work of Li et al in \cite{li2003time} we could weight more heavily documents which were created on (or near) this focus time.

\subsection{Complex Query Generation}\label{proposedapproach.complexquery}
With an approach proposed for extraction of temporal entities and named entities it is important to consider what other information in documents could be used to form a query. Considering the results from the L4 Project, especially the good performance of the \textit{All Terms Query} it seems important to consider other n-grams from the text which do not fall into the temporal or named entity category. These n-grams could be noun phrases or verb phrases or any other piece of text which provides additional context for a query.
For example from the text ``Hillary Clinton was today campaigning to become the democratic presidential nominee.'' we can extract the named entities. \textit{Hillary Clinton} is and \textit{democratic} is an organisation.
This lacks much of the context from the original text. We would like to know \textit{presidential nominee} and \textit{campaign} was included in the text as this potentially greatly narrows down the search results. The Stanford Core NLP toolkit as mentioned previously has the facility to do this using its parser. The parser identifies parts of speech tags allowing one to extract noun and verb phrases as well as other common constructions.

Having all of this information is not necessarily helpful unless we can formulate it into an effective query. Since the query is composed of several distinct classes of information it seems reasonable to formulate the query to represent this. The Indri Query Language (IndriQL) is an example technology which allows this kind of complex query formulation. A prototype version of Terrier v.5 was provided for this proposed project which contains some of the features of the IndriQL. With IndriQL and the new version of Terrier we can split a query into subqueries and also apply some additional complex rules inside the query. Some of these include the ability to match phrases inside arbitrary windows of text, either in order or unordered and to set synonyms for words which represent the same concept. Although the original IndriQL provides a facility to weight sections of the query as more or less important, the supplied version of Terrier does not support this. However, we can achieve the same effect through careful use of learning to rank by identifying appropriate features as we will discuss in the next section.

\paragraph{Text}
AP880507-0001
The government plans to start administering
drug tests to applicants for sensitive federal jobs and will refuse
employment to all who don't have confirmed negative tests,
according to a published report Saturday.
   ``This procedure will have a positive effect on reducing
instances of illegal drug use by employees ... and will provide for
a safer work environment,'' the Office of Personnel Management said
in a statement quoted in The Washington Post.
   Under the OPM plan based on plans submitted by dozens of
federal agency to Congress this week applicants about to be hired
for 345,528 sensitive positions will have to undergo urinalysis
tests with less than 48 hours notice, the newspaper said. The tests
will cover at least marijuana and cocaine.
\paragraph Headline
Government to Test Job Applicants for Drugs

\paragraph{Named Entities}
Washington Post
Congress
Office Personnel
Personnel Management
OPM

\paragraph{Temporal Entities}
19880507

\paragraph{Other Chunks}
government
administer
drug
tests
sensitive
federal
jobs
refuse
employment
confirmed
negative
procedure
positive reducing instance illegal employees safer environment statement based plans submitted dozens sensitive positions undergo urinalysis hours notice newspaper marijuana cocaine

\subsection{Learning to Rank} \label{proposed_approach.l2r}
As discussed in Section~\ref{background_survey.l2r} learning to rank is an information retrieval technique which has been applied with great success in many scenarios.
By using learning to rank we can not only tweak our retrieval system accordingly to achieve the best performance, but we can measure the importance of various features.
In order to begin to answer \textbf{RQ1} we must identify features which correspond to the use of each individual class of extracted information (named entities, temporal entities, other chunks, etc.).
The idea here being the after the learning to rank process a weight will be applied to each part of the query.
By considering a set of query dependent features which are functions of the number of matches of a given class of extracted information, we can weight these features accordingly.
The key challenge is here is to define appropriate features which represent this idea in order to acheive the effect of weighted query sections. We will investigate this by defining various features and applying learning to rank to determine the effectiveness of various feature definitions. Not only will this assist the query performance (in theory), it will provide quantitative data regarding the effectiveness of the distinct sections of the complex query. This data will be the learned weights of our defined features. We can apply this data to attempt to answer \textbf{RQ1}.
There is indeed more we can do with learning to rank.
Often other types of features are employed which do not depend on the query at all. 
These are called query independent features.
It will be worthwhile to investigate if the inclusion of certain query independent features (e.g. BM25 score or PageRank) have a positive effect on the learned model.
This relates to \textbf{RQ2} in that it seeks to supply additional techniques to improve the effectiveness of the IR system as a whole.

As mentioned in Section~\ref{background_survey.l2r} we propose the use the LambdaMART approach to learning to rank. The JForests implementation of this is included in Terrier and it has been shown to have state of the art performance.

Active learning as described by \cite{GeneratingQueriesLee12} is also relevant, however is beyond the scope of this work.
% \paragraph{Query Dependent Features}
% Features are expressed as subsqueries in the concrete implentation, however they can be thought of as function of the following:
% \begin{enumerate}[label=\textbf{F. \arabic*}]
% \item No. of exact matches of named entities from the query
% \item Number of exact matches of temporal entities from the query
% \item Number of matches in unordered windows of other noun phrase chunks.
% % \item Comparative document focus time (seperation of source and target document focus time}
% \end{enumerate}
% \paragraph{Query Independent}

\subsection{Knowledge Base Linking and Query Expansion}
Query expansion is another popular IR technique that was not touched upon in the L4 Project. 
This is the process of adding select terms to the query in order to improve it's performance. 
These terms can be added through pseudo-relevance feedback as discussed in Section~\ref{background_survey.knowledge_base}.
Another method is to use a knowledge base to discover more information on certain terms or entities. 
This means we can take a named entity like ``Barack Obama'' and use the knowledge base to provide additional terms like ``Barack Hussein Obama'' and ``44th President of the United States''. 
These can then be added to the query. on to include in queries well defined references to specific events or people.
This is an additional complex step in the query generation process and were it included it would need to be involved in the learning to rank process also. As such, this will not be attempted until all of the above has been implemented and evaluated.

\subsection{Overview of Approach}
The approach can be summarised in the following way.
First, we must build an index from the target collection. During this indexing step we must extract named entities, temporal entities and any other n-gram chunks we are interested in. 
We may decide that other information is to be extracted also later in the project.
We must organise the index in such a way that these specific classes of extracted information are easily accessible during retrieval.

Given a source document we must again analyse it, and extract the various classes of information.
These form a complex query using the parts of the IndriQL contained in Terrier.

These complex queries will be used in a custom learning to rank system which will learn weights of various features corresponding to each class of extracted information.

This learned model will be applied to queries generated from unseen source documents.

Fig. x demonstrates this flow of information in a graphical way.

\todo{include image}

\subsubsection{Full Worked Example}
This section gives a worked example of the various classes of information we wish to extract from source document in order to create complex queries.
Provided is a document which is representative of a source document.
\paragraph{Topic}
%Don't need the topic here, that's a different discussion
TREC Topic 51
Topic: Airbus Subsidies
Description: Document will discuss government assistance to Airbus Industrie, or mention a
trade dispute between Airbus and a U.S. aircraft producer over the issue of
subsidies. 

\paragraph{Full Text}

\begin{tabular}{|p{10cm}|p{3cm}|}
  This text will be wrapped & Some more text \\
\end{tabular}
Senior officials from Britain, France, West Germany and Spain on Tuesday ordered Europe's Airbus Industrie to try to reach an agreement on industrial and commercial cooperation with McDonnell Douglas Corp. by mid-1988.

The decision emerged at the end of a one-day meeting of transport ministers and chief executives of the industrial partners in the European commercial aircraft group.

Aviation industry observers said it was the latest display of Europe's eagerness to penetrate the lucrative U.S. market while trying to reduce trade friction over American claims of unfair competition by Airbus.

Trade negotiators from the United States, the European Community Commission and the four Airbus partner-countries are scheduled to meet in Konstanz, West Germany, March 18 to discuss U.S. claims that the Europeans unfairly subsidize the consortium.

French Transport Minister Jacques Douffiagues told reporters after the ministers' meeting that Airbus' mandate is ``unlimited,'' although the negotiations would tend to focus more on planned aircraft than on existing programs.

A cooperation accord with McDonnell could take the form of a joint venture or co-production scheme, he said.

Douffiagues suggested that possible areas of useful cooperation between Airbus and Mcdonnell could include small short-range jets with between 100-150 seats as well as large aircraft with 350 seats or more. ``There is no upper limit,'' he said.

``Any agreement with Mcdonnell must be balanced, taking account of the advantages and constraints for both sides,'' Douffiagues said.

Douffiagues said part of the interest in an agreement is that it would ease Airbus's access to the U.S. market, where the group has only three customers.

A statement released after the meeting said Airbus' mandate is to negotiate ``diligently'' and ``with a will to reaching a balanced agreement that is beneficial to the two parties.''

The industrial partners of the Airbus consortium \_ France's Aerospatiale, Britain's Aerospace, Messerschmitt-Boelkow-Blohm of West Germany and Spain's Construcciones Aeronauticas \_ have had contacts with Mcdonnell for almost two years now on eventual cooperation.

Previous talks with McDonnell Douglas on possible cooperation collapsed in 1986.

\paragraph{Named Entities}
We can extract the named entities from the above text as follows:
Locations
Britain
France
West Germany
Spain
Europe
U.S.
United States
Konstanz
West Germany
Construcciones Aeronauticas
\subparagraph{Persons}
Jacques Douffiagues
Douffiagues
Mcdonnell
\subparagraph{Orgs}
Airbus Industrie
McDonnell Douglas Corp.
Airbus
European Community Commission
McDonnell
Aerospatiale
Messerschmitt-Boelkow-Blohm
McDonnell Douglas

\paragraph{Temporal References}
We can also resolve the following specific temporal references from the text using Heideltime as discussed in Section~\ref{temporalcomparison}.
1988-03-01
1988
1988-03-18
1986

\paragraph{Other Chunks}
\todo{get the stanford parser to work!!!!}

\subsection{Evaluation and Experimental Set Up}
Having given an explanation of the proposed approach the implementation of this research will take we now explain how we can evaluate it's effectiveness.

\subsubsection{Hypotheses}
We have formed some hypotheses in response the research questions proposed in Section~\ref{problem_statement}. 
These hypotheses allows us to design and plan efficient experiments for evaluating the effectiveness of our proposed IR system. 
The hypotheses are formed in direct response to the research questions \textbf{RQ1} and \textbf{RQ2}.

\begin{enumerate}[label=\textbf{Hyp.\arabic*}]
\item Temporal Information will be less important to complex queries than named entities and other terms.
\item Adding context, like additional nouns and verbs other than named entities to queries will improve performance when there are many target documents which refer to the included set of named entities.
\end{enumerate}

\newcounter{nullhyp}
In response to \textbf{RQ1}:

\begin{enumerate}[label=\textbf{Null Hyp.\arabic*}]
\item Including temporal information will have no effect on query performance
\item Including other chunks, like noun phrases will have no effect on query performance.
\setcounter{nullhyp}{\value{enumi}}
\end{enumerate}

In response to \textbf{RQ2}:
\begin{enumerate}[label=\textbf{Null Hyp.\arabic*}]
\setcounter{enumi}{\value{nullhyp}}
\item Using learning to rank in order to weight the sections of the complex queries will have no effect compared to weighting all query sections equally.
\item We cannot efficiently and effectively extract information from source documents to formulate a complex query consisting of multiple classes of information.
\end{enumerate}

Having defined the null hypotheses we now define the hypotheses based on our observations from the related literature.

\subsubsection{Effectiveness Measures} \label{evalmeasures}
In order to evaluate the effectiveness of the retrieval system in relation to the above hypotheses we must employ formal measures of comparison.
Information retrieval is usually measures in terms of \textit{precision} and \textit{recall}. \textit{Precision} is the measure of how many of the retrieved documents are relevant, whereas \textit{recall} is a measure of how many of the total number of relevant documents were retrieved.
The most important and ubiquitous measure is likely \textit{Mean Average Precision} (MAP). 
MAP is commonly used in experiments involving TREC collections as it gives a good objective overview of the performance of given system and is well understood IR research communities.
\textit{Precision at 5/10} (P\@5/10) This measure allows us to measure the precision at the first n documents returned. 
That is, we can see a measure of how effective the system is a retrieving and ranking documents close to the top.
\textit{Mean Reciprocal Rank} (MRR) allows us to measure in a list of ranked results how close the top of the list the first relevant result was. The closer to the first rank the better.

In the L4 Project~\cite{DissertationKelvinFowler} the \textit{All Terms Query} was regarded as too slow to justify its use in an interactive user environment. 
This proposed research is a continued investigation into assisting the sensitivity review process at The National Archives so it is prudent to continue to measure query execution time. We must ensure that query results are returned in a reasonable time. %Justify the inclusion of this a little more

\subsubsection{Test Collection}
\todo{add table showing collection statistics}
For evaluation we use a collection of Associated Press articles from the TREC 1 Ad-Hoc task~\cite{trecnist}. The documents are a collection of Associated Press articles. These are indeed public domain documents and so are representative of the documents retrieved in the sensitivity review process. There are \textbf{\textit{79919}} documents in the target collection. These are a mixture of news articles. We will be using TREC Topics 51-100 from the TREC-1 Ad-Hoc task along with the supplied relevance judgements for TIPSTER disk 1\&2.

\subsubsection{Using Existing Relevance Judgements} \label{existingrelevance}
In order to ensure using existing topics and relevance judgements is a reasonable thing to do for this project this approach was prototyped.
Last years results were compared with results using the new test collection to ensure we retrieved comparable results. A detailed explanation can be seen below~\cite{NEEDED}
% Separate approach from implementation details

Drawing inspiration from Lee et al~\cite{GeneratingQueriesLee12} we propose to also use the collection of target documents as representative source documents. 
The generation of a test collection is an expensive task which requires many hours of manual work. 
In In order to do this we must look specifically at the existing topics and relevance judgements provided for the TREC 1 Ad-Hoc task. 
If we take a topic, we cant take a selected relevant document for this topic and use this document from which to generate queries. 
Provided the document is analogous in some way to the original query then the information need of ``public domain documents relating to this document'' is fulfilled by the relevant documents according to the qrels for the original topic.
We can form an effective test collection by choosing a number \code{N} of topics. 
For each of these topics we can choose a random relevant document according to the corresponding qrel entry. 
We must then check if the document is representative of the initial topic (as in it is representative of the description in the trec topics file). 
Then this document can be added to the source collection.
Note that this method means we must ignore the source document if it appears in the result set for a query, since it is likely to appear and must remain in the target collection for other queries.

In fact we can make this method even more comprehensive by rotating the document in use as a source document. Given a topic \code{T} and a set of relevant documents \code{D}, we can take every document $ d \in D $. We might assume that if our retrieval system is optimal creating a query from $ d $ would yield $ D \setminus \{d\} $. Thus, we can rotate the proxy source document and attempt to check if this is true. We will then have many more experiment to run, from which to analyse results.

\textbf{This approach to generating a test collection would not be a suitable substitution in other sensitivity review scenarios, but we are not considering document sentiment here, but rather the objective terms and phrases mentioned in a document, which are similar across source and target documents in practice (e.g. named entities, temporal references and noun phrase chunks).
There are some issues with this approach, such as the fact that it has not been specifically noted by archivists to represent the type of document up for review. Further, the target documents are very long in some cases. In the work of Lee et al~\cite{GeneratingQueriesLee12} they merely selected parts of a document to use a query which meant they could be more specific in choosing appropriately representative sections of text in order to make use of the existing relevance judgements. We propose to use this approach initially however we remain open to generating a manual test collection if the need arises.}

\subsection{Summary}
Thus, we have outlined a specific approach we propose to follow in order to implement and evaluate a solution to the proposed problem. We must effectively and efficiently extract relevant data from source documents. This data can be composed into a complex query to run in Terrier against the collection of target documents. We can use learning to rank to automatically weight the different sections of the query in a somewhat optimal way in order to improve performance on unseen documents. We can then evaluate this approach using existing documents collections as a test collection, in order to demonstrate the effectiveness of the new system.

%%%%%%%%%%%%%%%%%%%%%%%%%%%%%   WORK PLAN  %%%%%%%%%%%%%%%%%%%%%%%%%%%%%%%%%%%%%
\section{Work Plan} \label{work_plan}
\todo{work plan is very bare}
With the proposed approach detailed in the previous section we can describe some implementation details of this proposed approach and define some milestones of completion.
% With the above proposed approach in mind we can set some definitive milestones for this research project.

% With the proposed approach laid out above in Section~\ref{proposed_approach} we can now give a definitive description of the work that will need to be done in order to complete this research. This section will give an overview of the tasks that must be completed and will end with a Gantt Chart defining time scales for each section.


\todo{Try to give a more detailled explantation of the code you intent to Write}
% Theres gotta be an indexing phase, applied to both the source and target collections
% This indexing phase is going to have various steps, each of which you can describe in some detail.
% Then, depending on if it's source or target documents you actually have to something concrete with the info you extract.
% For source you'll be extracting and saving the individual bits in order to generate queries at a later date (might be a nice idea to allow different query formulations from one indexing pass)
% For target you need to add the terms to various (posting lists?????)
% Next is actual query generation, this isnt a textual thing to you can describe how you plan to pass the subquery components to terrier programatically.
% Then theres the actual retrieval, as in plumbing it all together to let terrier do it's thing.
% Then there's learning to rank where we let terrier figure out how to weight the different features (whatever these features might be, need to be clear about that)
% Evaluation next, create the test collection
% Do the evaluation using trec_eval in built into terrier
% Analyse the results
% Write up the dissy

% Theres very clear steps here, how do you go about seperating the ``proposed apporach'' to the work plan. Like where do we do the comparison of heideltime and sutime.
% Need to somehow effectively describe what a feature is....
\subsection{Query Component Extraction}
In order to formulate our complex queries as described in Section~\ref{proposedapproach.complexquery}, we must create a system which can read in a source document and return a collection of the complex query components. 
In order to extract the Named Entities we can continue to use StanfordNLP in a similar way to the L4 Project, with some slight adjustments.
This has been prototyped and we have successfully extracted n-gram named entities in order to retain multi word relationships.

We propose to use Heideltime for the extraction of temporal entities.
This works well with no major alterations and the times can be extracted programatically.
A particular challenge while indexing the target collection may be the identification of document creation times.
Confusingly, these exist as part of the Doc Id of each document and so some parser will need to be implemented to extract the dates from here.
This should be relatively simple.
It remains only to convert the Heideltime output into our desired format for storing temporal information.
Initially, at least, we wish to store temporal entities as a string so they can be directly compared to times mentioned in target documents during retieval.
Milliseconds since the epoch could be used for this, but this seems to be too granular for our needs.
Instead we propose simply to use YYYYMMDD format.

We also wish to extact other chunks from text in order to form queries.
This could be noun phrases or verb phrases or other constructions.
In order to do this we can again use the Stanford NLP toolkit.
It includes the Stanford Parser which annotates senstences with parts of speech tags.
We can identify different constructions using these parts of speech tags.

Each of these extraction steps will have to be implemented in a way which not only interacts well with indexing but can also be used in a standalone sense to create queries from documents.


\subsection{Complex Queries}
With these query components created we can now explicitly form our complex query, using the IndriQL features provided by Terrier v5. 
The inclusion of IndriQL features in Terrier is very new and there is little example code or documentation.
This will therefore require much experientation and investigation.
It may be required to implement some custom extensions to the IndriQL in Terrier in order to acheive the extact custom query formulation we wish to create.
For example, we may have to make new custom sizes of unordered windows. 
This query will not be provided to Terrier as merely text, but rather programmatically. 
Initially these queries will take a simple form of the various query components discussed above. 
However more query formulations can be created in due course.

Some combination of the extractors discussed above will be used in a class which receives a full document and outputs a complex query formulation.
This complex query will be modelled by a class in Java.

% This means we can match exactly bigram named entities as well as matching temporal expressions as unigrams. We can also match the other chunks. We can give each of these parts of the query a label. We should do this in several different configurations to ensure we are not missing any important query formulations.

\subsection{Learning to Rank}
With the creation of complex queries we will presumably have some sort of working retrieval system which can be evaluated in order to obtain baseline results.
In order to improve the performance and to obtain data relating to the performance of individual query sections we now wish to apply learning to rank.
Terrier provides learning to rank <> as standard.
We will be using the JForests implementation of LambdaMART.
Depending on the implementation of the above steps, we will produce a set of features, both query dependent and independent.
These features will be applied in learning to rank in different combinations.
The evalaution process must continue throughout this experimentation with feature combinations in order to discover the best configuration.
Specifically our set of features must include features which directly correlate to the use of section of a query, so that we may measure it's effectiveness and apply a ``weight'' to it.
We will not manually apply weights, instead the L2R process will achieve this for us.

Learning to rank relies on a sampling step, both during learning and application of the learned model.
Generally, a naive measure such as BM25 is used at this stage to achieve high recall.
The reranking step then uses the complex query with the defined features to reorder the results.
This initial sampling step will also require some investigation.
Although we may use a more basic retrieval model, a query is still required.
This may just be all the terms in the document with stopwords removed, although this will be slow.

The existing relevance judgements in the test collection can be used to train the model. 


% \subsection{Formulate Test Collection}
% We can use the process described in Section~\ref{existingrelevance} to generate our source collection from which to generate queries. Ideally this source collection would be fairly large in order to produce a good amount of data for evaluation.

\subsection{Evaluation}
Evaluation should begin as soon as there is a reasonable system upon which to evaluate. 
We will use Terrier's existing functionality to generate measurements of the suggested evaluation metrics discussed in Section~\cite{NEEDED}. 
The experiments will be crafted in direct response to the hypotheses from Section~\cite{NEEDED}.
With the source collection created we can run experiments using Terrier in order to produce results in the measures described in Section~\ref{evalmeasures}. 
These experiments will be run on each query configuration along with it's calculated weights after learning to rank.

\subsection{Further Work}
This work plan is flexible and allows the ability to add other features as we proceed. 
Once all of the above has been completed and evaluated we can look forwards to improving the query generation stage even more. Some potential avenues are query expansion through knowledge base linking or allowing more generality in temporal comparisons. We may also formulate and additional test collection in order to perform additional evaluation to more concretely rate the performance of our system.

\subsection{Time Frames}

% A table with rows containing a pbox of 0.17 textwidth, followed by 20 pboxes
% of 0.01 textwidth. Although this sums to just 0.37 textwidth, there are a
% lot of intercolumn gaps to consider. The 0.17 and 0.01 were fiddled by hand
% to fit this particular example.
\noindent\begin{tabular}{p{0.17\textwidth}
!{\vrule width 0.4mm}p{0.01\textwidth}*{3}{|p{0.01\textwidth}}
!{\vrule width 0.4mm}p{0.01\textwidth}*{3}{|p{0.01\textwidth}}
!{\vrule width 0.4mm}p{0.01\textwidth}*{3}{|p{0.01\textwidth}}
!{\vrule width 0.4mm}p{0.01\textwidth}*{3}{|p{0.01\textwidth}}
!{\vrule width 0.4mm}p{0.01\textwidth}*{3}{|p{0.01\textwidth}}
|}
% The top line
\textbf{Gantt chart} & \multicolumn{4}{c!{\vrule width 0.4mm}}{December} 
& \multicolumn{4}{c!{\vrule width 0.4mm}}{January}
& \multicolumn{4}{c!{\vrule width 0.4mm}}{February} 
& \multicolumn{4}{c!{\vrule width 0.4mm}}{March} 
& \multicolumn{4}{c|}{April} \\
% The second line, with its five years of four quarters
%\rpt[5]{& 1 & 2 & 3 & 4} \\
\hline
% using the on macro to fill in twenty cells as `on'
Query \mbox{Component} \mbox{Extraction}        \on[6] \off[14] \\
\hline
Learning to Rank   \off[4] \on[4] \off[4] \on[2] \off[6] \\
\hline
Evaluation using Test Collection    \off[5] \on[15]  \\
\hline
% using the on macro followed by the off macro
Other Query Formulations     \off[8] \on[4] \off[8]\\
\hline
% using the on macro followed by the off macro
Write-Up     \off[16] \on[4]\\
\hline
% The mbox prevent packages from being hyphenated
% The multicolumn produces no vertical guides within the columns it spans, but
% does put one at the end to complete the right-hand edge of the table
% \textbf{Work \mbox{packages}} & \multicolumn{20}{c|}{} \\
% \hline
% Finding Bugs  \on[2] \off[6] \on[2] \off[10] \\
% \hline
% Squashing Bugs \off[2] \on[4] \off[4] \on[1] \off[9] \\
% \hline
% % Note the omitting the count to on or off is the same as setting the count to 1
% Producing Results \off[6] \onx[13] \off \\
% \hline
% Dissemination \off[19] \on \\
% \hline
\end{tabular}

%%%%%%%%%%%%%%%%%%%%%%%%%%%%%%%%%%%%%%%%%%%%%%%%%%%%%%%%%%%%%%%%%%%
% it is fine to change the bibliography style if you want
\todo{Regularize the bib style}
\bibliographystyle{plain}
\bibliography{mprop}
\end{document}
